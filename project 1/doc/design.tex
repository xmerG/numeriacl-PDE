\documentclass[UTF8]{ctexart}
\usepackage{geometry}
\geometry{margin=1.5cm, vmargin={0pt,1cm}}
\setlength{\topmargin}{-1cm}
\setlength{\paperheight}{29.7cm}
\setlength{\textheight}{25.3cm}

% useful packages.
\usepackage{amsfonts}
\usepackage{amsmath}
\usepackage{amssymb}
\usepackage{amsthm}
\usepackage{enumerate}
\usepackage{graphicx}
\usepackage{multicol}
\usepackage{fancyhdr}
\usepackage{layout}
\usepackage{float, caption}
\usepackage{xcolor}
\usepackage{listings}
\usepackage{tikz}

% 自定义配色方案,尽量模仿 VS Code 的高亮效果
\definecolor{codegreen}{rgb}{0,0.6,0}
\definecolor{codeblue}{rgb}{0,0,0.9}
\definecolor{codepurple}{rgb}{0.58,0,0.82}
\definecolor{codered}{rgb}{0.8,0,0}
\definecolor{backcolor}{rgb}{0.95,0.95,0.95}

% lstlisting 的风格设置
\lstdefinestyle{vscode}{
	backgroundcolor=\color{backcolor},   % 背景颜色
	commentstyle=\color{codegreen},     % 注释颜色
	keywordstyle=\color{codeblue}\bfseries, % 关键字颜色
	numberstyle=\tiny\color{gray},      % 行号颜色
	stringstyle=\color{codered},        % 字符串颜色
	basicstyle=\ttfamily\footnotesize,  % 基本字体
	breakatwhitespace=false,            % 仅在空格处断行
	breaklines=true,                    % 自动换行
	captionpos=b,                       % 标题位置(bottom)
	keepspaces=true,                    % 保持空格
	numbers=left,                       % 显示行号
	numbersep=5pt,                      % 行号与代码间的间隔
	rulecolor=\color{black},            % 框线颜色
	showspaces=false,                   % 不显示空格符号
	showstringspaces=false,             % 不显示字符串中的空格
	showtabs=false,                     % 不显示制表符
	frame=single,                       % 外框
	tabsize=4,                          % 制表符宽度
	escapeinside={(*@}{@*)},            % 特殊字符转义
	morekeywords={*,...}                % 添加更多自定义关键字
}
\lstset{style=vscode}

% some common command
\newcommand{\dif}{\mathrm{d}}
\newcommand{\avg}[1]{\left\langle #1 \right\rangle}
\newcommand{\difFrac}[2]{\frac{\dif #1}{\dif #2}}
\newcommand{\pdfFrac}[2]{\frac{\partial #1}{\partial #2}}
\newcommand{\OFL}{\mathrm{OFL}}
\newcommand{\UFL}{\mathrm{UFL}}
\newcommand{\fl}{\mathrm{fl}}
\newcommand{\op}{\odot}
\newcommand{\Eabs}{E_{\mathrm{abs}}}
\newcommand{\Erel}{E_{\mathrm{rel}}}

\lstdefinestyle{json}{
	basicstyle=\ttfamily\footnotesize,
	commentstyle=\color{gray},
	stringstyle=\color{red},
	keywordstyle=\color{blue},
	numbers=left,
	numberstyle=\tiny\color{gray},
	stepnumber=1,
	numbersep=5pt,
	backgroundcolor=\color{white},
	showspaces=false,
	showstringspaces=false,
	showtabs=false,
	tabsize=2,
	breaklines=true,
	frame=single,
	captionpos=b
}

\begin{document}
	
	\pagestyle{fancy}
	\fancyhead{}
	\lhead{高凌溪, 3210105373}
	\chead{微分方程数值解project \#1 设计文档}
	\rhead{\today}
	\begin{abstract}
		本次编程作业设计了一个二维Poission方程($\Delta u=f$)求解器,用来求解定义在正方形区域$\Omega$和不规则区域$\Omega \setminus \mathbb{D}$的Poission的边值问题,实现了对Dirichlet边界条件、Neumann边界条件和混合边界条件三种边界条件对应的Poission方程求解。
	\end{abstract}
	\section{程序实现}
	\subsection{类和成员}
	\begin{enumerate}
		\item  \textbf{\texttt{EquationSolver}类}	首先我定义了\texttt{EquationSolver}类,用来实现求解Poission方程,私有成员如下
		\begin{lstlisting}[language=C++, caption={运算符实现}, label={lst:operators}]
		class EquationSolver{
		private:
			vector<vector<double>> grids;  // denote the 2 dimension grids
			vector<double> values; //求解的数值解将储存在values中
			int N=0;  //the number of grids in 1 dimension
			double h=0.0;
			Circle *c=nullptr;
			vector<bool> incircle;
			vector<double> laplacian;
			vector<vector<double>> A; 离散拉普拉斯算子得到的矩阵
			void precondition(); //处理内部的点
			void coeffMatrix(const vector<int> & mixed); //处理规则边界
			void Diri(const Function &g); //处理不规则边界的Dirichlet边值
			void Neum(const Function &g, const vector<int> &mixed);
			//处理不规则边界的Neumann边值
			void coeffMatrix(const Function &g, const vector<int> &mixed); //处理不规则边界
			void getcolumn(const Function &g, const vector<int> &mixed);
			//处理右端项
			void solve(const Function &g, const vector<int> &mixed);//lapacke求解
			vector<double> convert(const Function &g, const vector<int> &mixed);
			void adjust(const double &initial); //对不规则区域去掉圆内的格点,对Neumann边界条件加一个常数
			vector<double> errors(const Function &f) //计算误差
			vector<double> realValues(const Function &f) //计算真实值
		\end{lstlisting}
		公有成员如下,其中调用\texttt{void solveEquation(const Function \&g, const double \&initial=0.0, const vector<int> \&mixed=vector<int>{0,0,0,0,0})}函数将求解方程,这里的\texttt{const Function \&g}是边值条件,\texttt{const double \&initial=0.0}是点$(\frac{1}{h}, \frac{1}{h})$的值,由于Neumann边值问题的求解会相差一个常数,为了消除这个常数的影响,由用户自行指定希望得到的函数在$(\frac{1}{h}, \frac{1}{h})$的值。\texttt{const vector<int> \&mixed=vector<int>{0,0,0,0,0}} 给出混合边值条件,具体地:分别给直线$y=0,\ x=0,\ x=1,\ y=1,\ $编号为$0,\ 1,\ 2,\ 3,\ $,给$\partial \mathbb{D}$ 编号为$4$,$0$代表Dirichlet边值条件,$1$代表Neumann边值条件。
		\begin{lstlisting}[language=C++, caption={运算符实现},
			 label={lst:operators}]
		public:
			EquationSolver(const int &_N, const Function &f) //对于规则区域求解,需要输入格点数目和方程右端函数
			EquationSolver(const int &_N, const Function &f, Circle *_c):N(_N), c(_c) //对于不规则区域求解,需要输入格点数目和方程右端函数以及圆的参数
			void norm_error(const Function &f,const string &filename)//计算$l_1$ $l_2$ $l_\infty$范数下的误差,将误差输出到屏幕和文件
			void solveEquation(const Function &g, const double &initial=0.0, const vector<int> &mixed=vector<int>{0,0,0,0,0}) //求解方程
			void print(const string &filename, const Function &f) //将格点 格点上的真实值 格点上的数值解输出到文件。
		\end{lstlisting}
		\item \textbf{\texttt{Circle}类:}定义了\texttt{Circle}类,记录圆的圆心半径,实现了和一些简单的计算,如点到圆的距离,$x$轴方向点到圆的有向距离,$y$轴方向点到圆的有向距离。具体如下:
		\begin{lstlisting}[language=C++, caption={运算符实现},
			label={lst:operators}]
			class Circle{
				private:
				double x0=0.0;
				double y0=0.0;
				double radius=0.0;
				public:
				Circle();
				Circle(double x, double y, double r);
				bool inCircle(double x, double y) const;
				
				bool onCircle(double x, double y) const;
				
				double get_radius() const;
				
				double getX() const;
				
				double getY()const;
				
				double distance(double x, double y) const;
				
				double x_distance_to_circle(double x, double y) const;  //有向距离
				
				double y_distance_to_circle(double x, double y) const; //有向距离
				
				double angle_x_direction(double x) const; 
				
				double angle_y_direction(double y) const;
			};
			
		\end{lstlisting}
	\end{enumerate}
	\subsection{测试函数}
	测试函数全部定义在\texttt{TestFunction.cpp}中。一共定义了3个测试函数。
	\subsection{输入和输出}
	由于测试用例很多,采用\texttt{jsoncpp}控制输入和输出。
	\begin{itemize}
		\item  \textbf{输入:}输入文件放在\texttt{input}文件夹下,输入文件格式如下:
		\begin{lstlisting}[style=json]
			{
				"boundary_condition": "Mixed",
				"domain": "irregular",
				"grid_number": 32,
				"functionType": 3,
				"circle": [0.6,0.53,0.17],
				"mixed": [1,0,0,1,1]
			}
		\end{lstlisting}
		这里\texttt{"circle"}的参数分别是$x$坐标,$y$坐标,半径。\texttt{"mixed"}的参数只能是$0,\ 1$,代表边界上的Dirichlet条件或者Neumann边界条件。
		\item \textbf{输出:}最终程序运行的具体结果写在\texttt{output}文件夹下的\texttt{output.json}文件中,记录了格点、格点上数值解 、格点上的函数真实值。\texttt{error.json}格式如下:
				\begin{lstlisting}[style=json]
	{
		"boundary_condition": 0,
		"domain": 0,
		"l1_norm": 0.0018029833955581775,
		"l2_norm": 0.0008034791522725579,
		"linfinity_norm": 0.0005369806601569493
	}
			}
		\end{lstlisting}
	\end{itemize}

	\end{document}
