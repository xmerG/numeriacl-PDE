\documentclass[UTF8]{ctexart}
\usepackage{geometry}
\geometry{margin=1.5cm, vmargin={0pt,1cm}}
\setlength{\topmargin}{-1cm}
\setlength{\paperheight}{29.7cm}
\setlength{\textheight}{25.3cm}

% useful packages.
\usepackage{amsfonts}
\usepackage{amsmath}
\usepackage{amssymb}
\usepackage{amsthm}
\usepackage{enumerate}
\usepackage{graphicx}
\usepackage{multicol}
\usepackage{fancyhdr}
\usepackage{layout}
\usepackage{float, caption}
\usepackage{xcolor}
\usepackage{listings}
\usepackage{tikz}

% 自定义配色方案,尽量模仿 VS Code 的高亮效果
\definecolor{codegreen}{rgb}{0,0.6,0}
\definecolor{codeblue}{rgb}{0,0,0.9}
\definecolor{codepurple}{rgb}{0.58,0,0.82}
\definecolor{codered}{rgb}{0.8,0,0}
\definecolor{backcolor}{rgb}{0.95,0.95,0.95}

% lstlisting 的风格设置
\lstdefinestyle{vscode}{
	backgroundcolor=\color{backcolor},   % 背景颜色
	commentstyle=\color{codegreen},     % 注释颜色
	keywordstyle=\color{codeblue}\bfseries, % 关键字颜色
	numberstyle=\tiny\color{gray},      % 行号颜色
	stringstyle=\color{codered},        % 字符串颜色
	basicstyle=\ttfamily\footnotesize,  % 基本字体
	breakatwhitespace=false,            % 仅在空格处断行
	breaklines=true,                    % 自动换行
	captionpos=b,                       % 标题位置(bottom)
	keepspaces=true,                    % 保持空格
	numbers=left,                       % 显示行号
	numbersep=5pt,                      % 行号与代码间的间隔
	rulecolor=\color{black},            % 框线颜色
	showspaces=false,                   % 不显示空格符号
	showstringspaces=false,             % 不显示字符串中的空格
	showtabs=false,                     % 不显示制表符
	frame=single,                       % 外框
	tabsize=4,                          % 制表符宽度
	escapeinside={(*@}{@*)},            % 特殊字符转义
	morekeywords={*,...}                % 添加更多自定义关键字
}
\lstset{style=vscode}

% some common command
\newcommand{\dif}{\mathrm{d}}
\newcommand{\avg}[1]{\left\langle #1 \right\rangle}
\newcommand{\difFrac}[2]{\frac{\dif #1}{\dif #2}}
\newcommand{\pdfFrac}[2]{\frac{\partial #1}{\partial #2}}
\newcommand{\OFL}{\mathrm{OFL}}
\newcommand{\UFL}{\mathrm{UFL}}
\newcommand{\fl}{\mathrm{fl}}
\newcommand{\op}{\odot}
\newcommand{\Eabs}{E_{\mathrm{abs}}}
\newcommand{\Erel}{E_{\mathrm{rel}}}

\lstdefinestyle{json}{
	basicstyle=\ttfamily\footnotesize,
	commentstyle=\color{gray},
	stringstyle=\color{red},
	keywordstyle=\color{blue},
	numbers=left,
	numberstyle=\tiny\color{gray},
	stepnumber=1,
	numbersep=5pt,
	backgroundcolor=\color{white},
	showspaces=false,
	showstringspaces=false,
	showtabs=false,
	tabsize=2,
	breaklines=true,
	frame=single,
	captionpos=b
}

\begin{document}
	
	\pagestyle{fancy}
	\fancyhead{}
	\lhead{高凌溪, 3210105373}
	\chead{微分方程数值解project \#1 设计文档}
	\rhead{\today}
	\begin{abstract}
		本次编程作业设计了多重网格求解poisson方程。自己设计了限制算子,插值算子,求解Dircihlet边界、Neumann边界、混合边界条件的Poisson方程。
	\end{abstract}
	\section{程序实现}
	\subsection{类和成员}
	\begin{enumerate}
		\item  \textbf{\texttt{Sparse\_Matrix}类}	首先我定义了\texttt{Sparse\_Matrix}类,用来实现求解Poission方程,私有成员如下
		\begin{lstlisting}[language=C++, caption={运算符实现}, label={lst:operators}]
		class Sparse_Matrix{
			private: 
			vector<label> elements;
			int n;  //size of matrix
			public:
			Sparse_Matrix();
			Sparse_Matrix(const int &_n);
			Sparse_Matrix(const int &_n, const vector<label> &e);
			//Sparse_Matrix(Sparse_Matrix&& other) noexcept;
			//Sparse_Matrix& operator=(Sparse_Matrix&& other) noexcept;
			void setValues(const int &i, const int &j, const double &value); //set A(i,j)
			double operator()(const int &i, const int &j) const;
			Sparse_Matrix operator+(const Sparse_Matrix &B) const;
			Sparse_Matrix operator*(const Sparse_Matrix &B) const;
			Sparse_Matrix operator*(const double &a) const;
			Vector operator*(const Vector &v) const;
			vector<double> convert_to_vector() const;
			void Gauss_Seidel(Vector &initial, const Vector &b);
			int getdim() const;
			void print();
			
		};		
		\end{lstlisting}
		其中使用了稀疏矩阵的方法存储矩阵,只记录非零元素和非零元素的位置。并且实现了矩阵的移动构造、矩阵加法减法,矩阵的数乘,矩阵和矩阵、矩阵和向量的乘法以及Gauss\_Sediel法求解$Ax=b$。最后还实现了类似\texttt{matlab}按照行列标号取出矩阵的元素,如$A(i,j)$取出矩阵的第$(i, j)$个元素。
		\item  \textbf{\texttt{Vector}类:}	
		定义了\texttt{Vector}类,用来储存向量,具体的成员变量和函数如下:
		\begin{lstlisting}[language=C++, caption={运算符实现},
			 label={lst:operators}]
			class Vector{
				private:
				int n=0;  //length
				int m=0; //sqrt(n)
				vector<double> elements;
				public:
				Vector();
				Vector(const int &n);
				Vector(const vector<double> &e);
				Vector(const Vector& other);
				Vector(const int &n, const vector<double> &_e);
				Vector(Vector&& other) noexcept;
				void set_Value(const int &i, const double &value);
				void set_Value(const int &i, const int &j, const double &value);
				Vector& operator=(Vector&& other) noexcept;
				double operator()(const int &i) const;
				Vector operator+(const Vector &v) const;
				Vector operator-(const Vector &v) const;
				Vector operator*(const double &a) const;
				int getdim() const;
				void print() const;
				double operator()(const int &i, const int &j) const;
				void go_zero(const int &k);
				friend class Sparse_Matrix;
				vector<double> getelements() const;
				double infinity_norm() const;
				double l2_norm() const;
				double l1_norm() const;
				void projection(); //投影到与kernel正交的子空间
				void copy(const Vector& other);
				void print_to_file(const string &filename);
			};
		\end{lstlisting}
		实现了向量的移动构造函数、向量的赋值(避免拷贝)向量的加法、减法,计算向量的范数。对于二维的Poisson问题对应的向量,可以视为一维向量对应的张量积,也支持$v(i,j)$的操作取出网格$(ih,jh)$上的元素。
		\item \textbf{限制算子和插值算子}
		具体的实现见\texttt{report.tex},保存在\texttt{src}下的\texttt{restriction}和\texttt{prolongation}文件夹下。
		\item  \textbf{ Multigrid 类}
		\begin{lstlisting}[language=C++, caption={运算符实现},
			label={lst:operators}]
				template<int dim>
				class Multigrid{
					private:
					map<int, couplet> discretors; //int 代表网格数目,如2代表2的网格数,4代表4的网格数 
					int n;  //网格个数
					Vector solutions;
					BoundaryCondition BC;
					int counter=1; //迭代次数 
					string C;
					unique_ptr<Retriction<dim>> restriction;
					unique_ptr<Prolongation<dim>> prolongation;
					Vector w_Jacobi(int i, const Vector &initial);
					Vector V_cycle(const int &n, Vector& initial_guess, int nu1, int nu2);
					Vector FMG(const int &_n, int nu1, int nu2);
					vector<double> corsa_solve(const int &i);
					void create_grids_D(const Function &f, const Function &g,const int &i);
					void create_grids_N(const Function &f, const Function &g,const int &i);
					void create_grids_M(const Function &f, const Function &g,const int &i, 
					const vector<int> &mixed=vector<int> {0,0,0,0});
					Vector error(const Function &f);
					
					public:
					Multigrid();
					Multigrid(const Function &f, const Function &g, BoundaryCondition bc, const int &i,
					const vector<int> &mixed=vector<int>{0,0,0,0});
					void solve(const string &restriction, const string &prolongation, const string &cycle, Vector& initial_guess, int nu1, int nu2,
					double tol,const double &value=0.0, int max_itr=50);
					void print();
					void print_to_file(const string &filename, const Function &f);
					void print_solution(const string &filename) const;
				};
			
		\end{lstlisting}
		本次作业的核心。实现了FMG和V-cylce两种方法,支持用户输入$-\delta u=f$中的右端项$f$,边界条件,网格数目,限制算子,插值算子,初值, 相对误差,最大迭代次数。如果时Neumann边值条件,应当给出原函数在$(0,0)$上的值。如果是$Mixed$边界条件,还应额外给出在边界上的具体边界条件,用\texttt{vector<int>}给出边界条件,$0$代表Dirichlet边界条件,$1$代表Neumann边界条件。其中函数\texttt{w\_Jacobi}实现松弛操作,之后在$n=4$的网格上用Gauss-Sediel迭代求出真实解。函数\texttt{FMG} \texttt{V-cycle}分别实现FMG和v-cycle。
	\end{enumerate}
	\subsection{测试函数}
	测试函数全部定义在\texttt{testFunction.h}中。
	\subsection{输入和输出}
	由于测试用例很多,采用\texttt{jsoncpp}控制输入和输出。
	\begin{itemize}
		\item  \textbf{输入:}输入文件放在\texttt{input}文件夹下,输入文件格式如下:
		\begin{lstlisting}[style=json]
			  {
				"dimension" :1,
				"grid_number" :256,
				"boundary_conditions":"Mixed",
				"mixed":[0,1],
				"restriction_operator": "full_weighting",  
				"interpolation_operator": "linear",        
				"cycle_type": "FMG",                 
				"max_iterations": 50,
				"relative_accuracy": 1e-8,
				"initial_guess": "zero"                
			},
		\end{lstlisting}
		\item \textbf{输出:}最终程序运行的具体结果写在\texttt{output}文件夹下的\texttt{output.json}文件中,格式如下:
		\begin{lstlisting}[style=json]
			    {
				"boundary_condition": 1,
				"cycle": "FMG",
				"dimension": 1,
				"infinity_norm": 9.397917202669248e-06,
				"iterator times": 5,
				"l1_norm": 0.0016597179004913531,
				"l2_norm": 0.00011345851360180907,
				"number": 256
			},
			
		\end{lstlisting}
		由于本次测试数据规模很大,没有输出真实解和数值解,只输出了无穷范数和$l_1 norm,\ l_2 norm$来判断求解器的效果。
	\end{itemize}

	\end{document}
